\begin{parag}{Objectifs du cours}
    Voici ici une liste du gros des choses à savoir pour les examens d'hiver 2025
\end{parag}
\begin{parag}{Méthode de Gauss: échelonner}
    \begin{itemize}
        \item Système incompatible
        \item Aucune, une ou une infinité de solutions
        \item Déterminer la dimension de la solution générale
        \item Calculer le rang d'une application linéaire
        \item Calculer la dimension du noyau
        \item Inverser une matrice
    \end{itemize}
\end{parag}
\begin{parag}{Opération sur lignes/colonnes}
    \begin{itemize}
        \item Calculer le déterminant
        \item Calculer l'aire/volume d'une forme en se basant sur le déterminant
        \item \textcolor{orange}{(}Factorisation $LU$ \textcolor{orange}{)}
        \item Extraire une base
        \item Compléter une base
    \end{itemize}
\end{parag}
\begin{parag}{Application linéaire}
    \begin{itemize}
        \item Matrice de changement de base
        \item  théorème du rang
        \item injéctivité / surjectivité
        \item Reconnaître des sous-espaces vectoriels de $\mathbb{R}^n, \mathbb{P}_n, M_{m \times n}(\mathbb{R})$
        \item Travailler avec les sous-espaces, noyau, etc$\dots$
    \end{itemize}
\end{parag}
\begin{parag}{Diagonalisation}
    \begin{itemize}
        \item Calculer le polynôme caractéristique
        \item Calculer les valeurs propres (réelles ou complexes) et les espaces propres (pas pour les valeurs complexes)
        \item Trouver une base $\bmath$ de vecteur propres
        \item Ecrire la matrice de changement de base
    \end{itemize}
\end{parag}
\begin{parag}{Gram-Schmidt}
    \begin{itemize}
        \item Produit scalaire standard, norme et orthogonalité
        \item Formule pour la projection orthogonale (pour une base orthogonale ou orthonormée)
        \item Formule pour Gram schmidt
        \item Méthode des moindres carrées
        \item Factorisation $QR$
        \item Droite de régression linéaire
    \end{itemize}
\end{parag}
\begin{parag}{Orthodiagonalisation}
    \begin{itemize}
        \item Matrice orthogonales
        \item Matrice symétrique
        \item Critère d'orthodiagonalisation
        \item Théorème spectrale
    \end{itemize}
\end{parag}
\begin{parag}{Propre aux cours}
    \begin{itemize}
        \item \textcolor{orange}{(} Formule de Cramer \textcolor{orange}{)}
        \item \textcolor{orange}{(} produit scalaire non standard \textcolor{orange}{)}
        \item \textcolor{orange}{(} Décomposition en valeur singulière\textcolor{orange}{)}
        \item \textcolor{orange}{(}Interprétation du théorème spectrale \textcolor{orange}{)}
        \item Corps finis:
        \begin{itemize}
            \item Les corps $\mathbb{F}_p$ des entiers modulo $p$
            \item Construction de $\mathbb{F}_{p^2}$ et $\mathbb{F}_{p^3}$
            \item Calcul de produit et d'inverse dans $\mathbb{F}_{p^2}$ et $\mathbb{F}_{p^3}$
            \item Algèbre linéaire sur $\mathbb{F}_p$ et $\mathbb{F}_{p^2}$
        \end{itemize}
    \end{itemize}
\end{parag}
\begin{parag}{Note pour l'examen}
    L'examen est fixé à $80$ points et 180 minutes ce qui donne $\approx 2$ minutes par point. Donc si pour un vrai ou faux on commence à faire une page entière de justification, c'est souvent qu'il y a plus simple.
\end{parag}