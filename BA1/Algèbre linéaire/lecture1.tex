\lecture{1}{2024-09-10}{Equation linéaire}{}
\begin{parag}{Equation linéaires}
    \textbf{Notation}
    \begin{itemize}
        \item Les lettres $k, m, n, \dots$ représentent des nombres entiers
        \item Les lettres $a, b, c\dots$ ou $a_1, a_2, \dots, a_n$ seront utilisées pour des nombres (réels), des \important{paramètres}
        \item Les lettres $x, y, z$ ou $x_1, x_2, \dots, x_n$ seront utilisées pour les \important{inconnues}
    \end{itemize}

    On note \R pour les nombres réels, qu'on représente par une droite.
    \begin{itemize}
        \item $\mathbb{R}^2 = \mathbb{R} \times \mathbb{R} = \{(a, b) | a, b, \in \mathbb{R}\}$ est le plan (cartésien)
        \item $\mathbb{R}^n = \{(a_1, \dots, a_n)|a_i \in \mathbb{R}^n\}$
        \item En général $X \times Y = \{(x, y)| x \in X, y \in Y\}$
    \end{itemize}
\end{parag}
\begin{parag}{Matrice échelonnée et réduite}
    \begin{definition}
        Une matrice est échelonnée et \important{réduite} si
        \begin{itemize}
            \item les lignes non nulles se trouvent au--dessus des lignes nulles;
            \item le premier coefficient non nul d'une ligne (le coefficient principal) se trouve à droite du coefficient principal des lignes précédentes
            \item les coefficients situées en-dessous d'un coefficient principal des lignes précédentes
            \item les coefficients situés en-dessous d'un coefficient principal sont nuls
            \\
            Elle est dire \important{échelonnée-réduite} si de plus
            \item les coefficients principaux sont tous égaux à 1
            \item  dans la colonnes d'un coefficient principal tout les autres coefficients sont nul
        \end{itemize}
    \end{definition}
\end{parag}

\begin{parag}{Pivots et comptabilité}
    \begin{definition}
        Les emplacements des coefficients principaux sont des \important{pivots} (ou échelons). Les colonnes de ces coefficients sont les colonnes pivots.
    \end{definition}
    \begin{definition}
        Un système d'équation linéaire est \important{compatible} s'il admet au moins une solution. Sinon il est dit incompatible
    \end{definition}
    S'il y a un pivot dans la dernière colonne (celle des termes inhomogènes), on a alors une équation 0 = $\bar{b}_{k+1}$ et $\bar{b}_{k+1}$ est non nul. Il n'y a donc aucune solution
\end{parag}